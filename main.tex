\documentclass[a4paper]{article}

\usepackage[usenames,dvipsnames]{color}
\usepackage[
  colorlinks = true,
  linkcolor  = Maroon,
  urlcolor   = MidnightBlue,
  citecolor  = ForestGreen
  ]{hyperref}
\usepackage{natbib}
\usepackage{ifthen}
\renewcommand{\hscodestyle}{\small}


% use Wouter's setup for notes (extended with ``hidden'')
\newboolean{marginNotes}
\setboolean{marginNotes}{false}

\newboolean{showNotes}
\setboolean{showNotes}{true}

\newboolean{showHidden}
\setboolean{showHidden}{false}

\newcommand{\hidden}[1]{
  \ifthenelse{\boolean{showHidden}}
             {#1}
             {}}
\newcommand{\marginNote}[1]{
  \ifthenelse{\boolean{marginNotes}}
             {\marginpar{#1}}
             {#1}}
\newcommand{\todo}[1]{
  \ifthenelse{\boolean{showNotes}}
             {\marginNote{\textcolor{red}{\textbf{TODO:~}#1}}}
             {}}

\renewcommand{\sectionautorefname}{\S}
\renewcommand{\subsectionautorefname}{\S\S}

% use Buss' Proofs
\usepackage{mdframed}
\usepackage{bussproofs}
\EnableBpAbbreviations%
\def\fCenter{\ \vdash\ }
\def\limpl{\multimap}
\newenvironment{scprooftree}[1]%
  {\gdef\scalefactor{#1}\begin{center}\proofSkipAmount \leavevmode}%
  {\scalebox{\scalefactor}{\DisplayProof}\proofSkipAmount \end{center} }

% define various which I'll be sure to use symbols
\usepackage{textgreek}
\usepackage{stmaryrd}
\usepackage{relsize}
\usepackage{ucs}
\usepackage[utf8x]{inputenc}
\DeclareUnicodeCharacter{8656}{$\slash$}
\DeclareUnicodeCharacter{8658}{$\backslash$}
\DeclareUnicodeCharacter{8666}{$\varoslash$}
\DeclareUnicodeCharacter{8667}{$\varobslash$}
\DeclareUnicodeCharacter{7478}{$^{J}$}
\DeclareUnicodeCharacter{7480}{$^{L}$}
\DeclareUnicodeCharacter{7486}{$^{P}$}
\DeclareUnicodeCharacter{7487}{$^{R}$}


% customise the appearance of itemise bullets
\def\myitem{\item[$\circ$]}


\begin{document}

\section{Introduction}

\subsection{What is type-logical grammar?}\label{sec:what-is-type-logical-grammar}
Before we address the question of what type-logical grammar is, let us
look at the abstract pipeline for natural language understanding
in~\autoref{fig:abstract-nlu-pipeline}.
%
\begin{figure}
  \begin{mdframed}
    \centering
    \begin{minipage}{0.35\linewidth}%
      \centering
      \framebox{Morphological}\\
      $\downarrow$\\
      \framebox{Lexical}\\
      $\downarrow$\\
      \framebox{Syntactic}\\
      $\downarrow$\\
      \framebox{Semantic}\\
      $\downarrow$\\
      \framebox{Pragmatic}\\
    \end{minipage}
    \begin{minipage}{0.55\linewidth}%
      \centering
      ``Mary saw foxes.''\\
      $\downarrow$\\
      Mary see.PAST fox.PL\\
      $\downarrow$\\
      Mary:NP see:TV.PAST fox:NP.PL\\
      $\downarrow$\\
      Mary:NP [see:TV.PAST fox:NP.PL]\\
      $\downarrow$\\
      $\exists X. X \subseteq\mathbf{fox}\land\mathbf{past}(\mathbf{see}(\text{Mary},X))$\\
      $\downarrow$\\
      \ldots\footnotemark\\
    \end{minipage}
  \end{mdframed}
  \caption{An abstract pipeline for natural language
    understanding.}%
  \label{fig:abstract-nlu-pipeline}
\end{figure}
\footnotetext{%
  As the pragmatic function is all about integrating context (be it
  textual or environmental) into the meaning, it would not make sense
  to list something here.
}
%%% Local Variables:
%%% mode: latex
%%% TeX-master: t
%%% End:
%
%
To the left of the figure, you see the various phases or functions
commonly associated with an NLU-pipeline. To the right, you see the
inputs and outputs of these functions.
For instance, the morphological function will take an unanalysed
sentence, and return a sentence which is lemmatised and has all
morphemes made explicit (in the case of the example
in~\autoref{fig:abstract-nlu-pipeline}, \texttt{PAST} and
\texttt{PLURAL}).

There is some disagreement on the exact role of type-logical grammars in
this pipeline. Ideally, type-logical grammars would play the role of
both the syntactic and the semantic function. However, the current
state of affairs in research is that often only the semantic function
is truly considered\footnote{%
  This statement is not true for \emph{associative} type-logical
  grammars, which fundamentally treat the syntactic and semantic
  functions as the same function by rejecting the tree structure of
  sentences.
}. That is the approach we will also take in this thesis -- that is,
we consider type-logical grammars to be the function:
%
\begin{center}
  Mary:NP [see:TV.PAST fox:NP.PL]\\
  $\downarrow$\\
  \framebox{Type-Logical Grammar}\\
  $\downarrow$\\
  $\exists X.X\subseteq\mathbf{fox}\land\mathbf{past}(\mathbf{see}(\text{mary},X))$
\end{center}
%
The problem that the semantic function is trying to address is the
following:
%
\begin{center}\itshape
  Given a tree of words, compute the meaning of the overall tree.
\end{center}
%
Type-logical grammar offers a solution to this problem. A type-logical
grammar consists of three things:
\begin{enumerate}[label=(\arabic*)]
\item a syntactic calculus, set up in such a fashion that only
  grammatical sentences are well-typed, and for which an efficient
  method of proof-search exists;
\item a semantic calculus, used to represent the meanings of words and
  sentences; and
\item a translation from the syntactic to the semantic calculus.
\end{enumerate}
We interpret the part-of-speech tags in our input as (NP, TV, etc.)
types in the syntactic calculus, and use these -- together with the
desired constituent type -- to form an input sequent. We then search
for a term of that type in the syntactic calculus, and translate it to
the semantic calculus. Once there, we interpret the morphemes
(e.g. lemmas, \texttt{PAST} and \texttt{PLURAL}) as (possibly complex)
terms in the semantic calculus.

In \autoref{sec:simple-type-logical-grammar}, we will have a look at
an example type-logical grammar, and give some examples of this
process of computing sentence meaning.

\subsection{A simple type-logical grammar}\label{sec:simple-type-logical-grammar}
The simplest type-logical grammar that comes to mind -- drawing
heavily from Montague grammar and categorial grammar -- is composed of
the simply-typed lambda calculus with atomic types \e\ and \t\ (\lamET) as
a semantic calculus, and the non-associative Lambek calculus (NL) as a
syntactic calculus.

The usual natural deduction formulation of \lamET\ can to be seen
in~\autoref{fig:implicit-semantic-calculus}. It is a simple lambda
calculus, with atomic types \e\ (`entity') and \t\ (`truth-value'). In
addition, all logical operators and word-meanings we need are assumed
to be defined as constants, with their expected type. For instance,
$\forall$ is a constant of type $(\e\ra\t)\ra\t$, and \john\ is a
constant of type \e. Note that we will often write logical operators
in their usual notation, e.g.\ $M\land N$ or $\forall x.M$, but this
should be take as syntactic sugar for constants, abstractions and
applications.

\begin{figure}
  \begin{mdframed}
    \centering
    \begin{alignat*}{4}
      \text{Type}       \;&A,B&&\coloneqq \e\vsep\t\vsep A\ra B\\
      \text{Term}       \;&M,N&&\coloneqq x\vsep C\vsep\lambda x.M\vsep(M\;N)\\
      \text{Constant}   \;&C  &&\coloneqq
      {\forall}\vsep{\exists}\vsep{\neg}\vsep{\supset}\vsep{\land}\vsep{\lor}\vsep\ldots\\
      \text{Environment}\;&Γ  &&\text{set of typing assumptions of the form `$M : A$'}
    \end{alignat*}

    \begin{pfbox}
      \AXC{$(x : A)\in Γ$} \RightLabel{{Ax}} \UIC{$Γ\fCenter x : A$}
    \end{pfbox}

    \vspace*{\baselineskip}
    \begin{pfbox}
      \AXC{$Γ,x : A\fCenter M : B$} \RightLabel{$\ra${I}}
      \UIC{$Γ\fCenter \lambda x.M : A\ra B$}
    \end{pfbox}
    \begin{pfbox}
      \AXC{$Γ\fCenter M : A\ra B$} \AXC{$Γ\fCenter N : A$}
      \RightLabel{$\ra${E}} \BIC{$Γ\fCenter (M\;N) : B$}
    \end{pfbox}

    \vspace*{\baselineskip}
  \end{mdframed}
  \caption{A simple semantic calculus.}%
  \label{fig:implicit-semantic-calculus}
\end{figure}
%
%%% Local Variables:
%%% mode: latex
%%% TeX-master: t
%%% End:
%

Using this calculus as a semantics function directly would
over-generate as, e.g. for the sequent $\{\john:\e,
\likes:\e\ra\e\ra\t, \mary:\e\}\fCenter\t$ we can derive
both $((\likes\;\john)\;\mary)$ and $((\likes\;\mary)\;\john)$.
The reason for this is, of course, that the set structure used in this
formulation is much too expressive for natural language grammar.

If we want more control over the structure of our terms, a good first
step is to move to a purely syntactic formulation, where all the
structural properties are made explicit in the calculus itself; this
has been done in \autoref{fig:explicit-semantic-calculus}. We have
replaced the set by a (possibly empty) binary tree, spanned by the
structural product `$\prod$', and have included a number of new
structural rules which implement the structure of a set.
% footnote:
%   In \autoref{fig:implicit-semantic-calculus}, we can use the
%   structure of a set because it is populated by \textit{typing
%   assumptions} of the form $x : A$ -- as such, the environment can
%   contain both $x : A$ and $y : A$. When we dropped the terms from the
%   formulation in \autoref{fig:explicit-semantic-calculus}, we had to
%   make the switch to multisets.

\begin{figure}
  \begin{mdframed}
    \centering
    \begin{minipage}{0.6\linewidth}
      \begin{alignat*}{4}
        \text{Atom}     \;&α    &&\coloneqq \e\vsep\t\\
        \text{Type}     \;&A,B  &&\coloneqq α\vsep A\ra B\\
        \text{Structure}\;&Γ,Δ,Π&&\coloneqq A\vsep\emptyset\vsep Γ\prod Δ\\
        \text{Context}  \;&Σ    &&\coloneqq \Box\vsep Σ\prodl Δ\vsep Γ\prodr Σ
      \end{alignat*}
    \end{minipage}%
    \begin{minipage}{0.4\linewidth}
      \begin{align*}
        \Box [Γ]&\mapsto Γ\\
        (Σ\prodl Δ)[Γ]&\mapsto (Σ[Γ]\prod Δ)\\
        (Δ\prodr Σ)[Γ]&\mapsto (Δ\prod Σ[Γ])
      \end{align*}
    \end{minipage}

    \vspace*{\baselineskip}
    \begin{pfbox}[0.9]
      \AXC{}\RightLabel{{Ax}}\UIC{$A\fCenter A$}
    \end{pfbox}

    \vspace*{\baselineskip}
    \begin{pfbox}[0.9]
      \AXC{$Γ\prod A\fCenter B$}
      \RightLabel{$\ra${I}}
      \UIC{$Γ\fCenter A\ra B$}
    \end{pfbox}
    \begin{pfbox}[0.9]
      \AXC{$Γ\fCenter A\ra B$}
      \AXC{$Δ\fCenter A$}
      \RightLabel{$\ra${E}}
      \BIC{$Γ\prod Δ\fCenter B$}
    \end{pfbox}

    \vspace*{\baselineskip}
    \begin{pfbox}[0.9]
      \AXC{$Σ[Γ\prod \emptyset]\fCenter B$}
      \RightLabel{$\emptyset${E}}
      \UIC{$Σ[Γ]\fCenter B$}
    \end{pfbox}

    \vspace*{\baselineskip}
    \begin{pfbox}[0.9]
      \AXC{$Σ[A\prod A]\fCenter B$}
      \RightLabel{Cont.}
      \UIC{$Σ[A]\fCenter B$}
    \end{pfbox}
    \begin{pfbox}[0.9]
      \AXC{$Σ[Γ]\fCenter B$}
      \RightLabel{Weak.}
      \UIC{$Σ[Γ\prod Δ]\fCenter B$}
    \end{pfbox}

    \vspace*{\baselineskip}
    \begin{pfbox}[0.9]
      \AXC{$Σ[Δ\prod Γ]\fCenter B$}
      \RightLabel{Comm.}
      \UIC{$Σ[Γ\prod Δ]\fCenter B$}
    \end{pfbox}
    \begin{pfbox}[0.9]
      \AXC{$Σ[(Γ\prod Δ)\prod Π]\fCenter B$}
      \doubleLine\RightLabel{Ass.}
      \UIC{$Σ[Γ\prod (Δ\prod Π)]\fCenter B$}
    \end{pfbox}
    \vspace*{\baselineskip}
  \end{mdframed}
  \caption{Explicit \lamET, a simple semantic calculus with explicit
    structural rules.}%
  \label{fig:explicit-semantic-calculus}
\end{figure}
%
%%% Local Variables:
%%% mode: latex
%%% TeX-master: t
%%% End:


Note that, in order to define these structural rules, we had to define
the notion of a `context' --  which is a structure with \emph{exactly
one} hole in it -- and a plugging function `\plug' -- which inserts a
structure into that hole. The reason for this is that we have to be
able to apply commutativity and associativity \emph{anywhere} in the
structure to be able to freely change the order (and bracketing) of
the formulas in it.\footnote{%
  The contexts are not strictly necessary for $\emptyset E$,
  contraction and weakening, since we can already move any formula
  anywhere we want, but they make the proof system much more usable
  and greatly decrease the length of proofs that need to use any of
  these structural rules.
}

We can use the term language from \autoref{fig:implicit-semantic-calculus}
for the explicit version of \lamET. The term labelling of the logical
rules is exactly the same. The structural rules only manipulate
structures, and therefore do not change the terms. The only exception
to this is contraction, for which the term labelling is as follows:
\begin{prooftree}
  \AXC{$Σ[y : A\prod z : A]\fCenter M : B$}
  \RightLabel{Cont.}
  \UIC{$Σ[x : A]\fCenter M[x/y][x/z] : B$}
\end{prooftree}
Contraction, therefore, contracts two names. For this it uses
substitution, which is defined as usual:
\begin{alignat*}{3}
  &x             &&[N/y] \mapsto
  \begin{cases}
    N, &\text{if}\;x=y\\
    x, &\text{otherwise}
  \end{cases}
  \\
  &C             &&[N/y] \mapsto C\\
  &(\lambda x.M) &&[N/y] \mapsto
  \begin{cases}
    \lambda x.M[N/y], &\text{if}\;x=y\\
    \lambda x.M,      &\text{otherwise}
  \end{cases}
  \\
  &(M\;M')       &&[N/y] \mapsto (M[N/y]\;M'[N/y])
\end{alignat*}

Now that we have a semantic calculus where all structure has been made
explicit, we can construct our syntactic calculus in three simple
steps:
\begin{enumerate}
\item%
  we drop \emph{all} structural rules;
\item%
  since the implication `$\ra$' can now only take arguments directly
  from the left, we add a second implication `$\la$' which can only
  take arguments from the right -- by convention, implications in this
  system are written as division lines, with the argument \emph{under}
  the line (so, as `$\impr$' and `$\impl$', respectively);
\item%
  we replace the atomic semantic types \e\ and \t\ by atomic syntactic
  types, reminiscent of part-of-speech tags -- in this case, these
  are S (`sentence'), NP (`noun phrase'), N (`noun'), PP
  (`prepositional phrase') and INF (`infinitive').
\end{enumerate}
The resulting system can be seen in \autoref{fig:syntactic-calculus},
defined along with the type-aliases \A\ (`adjective'), \IV\
(`intransitive verb') and \TV\ (`transitive verb').

\begin{figure}
  \begin{mdframed}
    \centering
    \begin{alignat*}{4}
      \text{Type}     \;&A,B&&\coloneqq\text{S}\vsep\text{N}\vsep\text{NP}\vsep\text{PP}\vsep \text{INF}\vsep A\impr B\vsep B\impl A\\
      \text{Structure}\;&Γ,Δ&&\coloneqq A\vsep Γ\prodΔ\\
    \end{alignat*}

    \begin{pfbox}
      \AXC{} \RightLabel{{Ax}} \UIC{$A\fCenter A$}
    \end{pfbox}

    \vspace*{\baselineskip}
    \begin{pfbox}
      \AXC{$A\prodΓ\fCenter B$} \RightLabel{$\impr${I}}
      \UIC{$Γ\fCenter A\impr B$}
    \end{pfbox}
    \begin{pfbox}
      \AXC{$Γ\fCenter A$} \AXC{$Δ\fCenter A\impr B$} \RightLabel{$\impr${E}}
      \BIC{$Γ\prodΔ\fCenter B$}
    \end{pfbox}

    \vspace*{\baselineskip}
    \begin{pfbox}
      \AXC{$Γ\prod A\fCenter B$} \RightLabel{$\impl${I}}
      \UIC{$Γ\fCenter B\impl A$}
    \end{pfbox}
    \begin{pfbox}
      \AXC{$Γ\fCenter B\impl A$} \AXC{$Δ\fCenter A$} \RightLabel{$\impl${E}}
      \BIC{$Γ\prod Δ\fCenter B$}
    \end{pfbox}

    \vspace*{\baselineskip}
  \end{mdframed}
  \caption{A simple syntactic calculus.}%
  \label{fig:syntactic-calculus}
\end{figure}
%
%%% Local Variables:
%%% mode: latex
%%% TeX-master: t
%%% End:


Dropping \emph{all} structural rules may seem unnecessary, but there
is a good motivation for each rule.  For example, in the presence of
commutativity, there is no way to distinguish between ``Mary walks.''
and ``walks Mary.''; under weakening, we can add any word anywhere in a
grammatical sentence, and the sentence will remain grammatical --
e.g. ``Mary banana walks.''; and with contraction, we can remove
consecutive words with the same type -- this means that ``John gave
Bill a book.'' would be just as grammatical as ``John gave Bill.'',
and the two could be taken to mean the same!

The motivations to drop associativity and the empty structure are a
little bit harder to understand, but the interested reader can refer
to \citet[p. 33, 105-106]{moot2012}, who show some ungrammatical
sentences that would be accepted in the presence of these rules.

The last component we need for a type-logical grammar is a translation
from our syntactic calculus to our semantic calculus. This translation
is described in \autoref{fig:translation-1}:
\begin{enumerate*}[label=(\arabic*)]
\item a function $\tr$, translating the types in NL to types in
  \lamET; and
\item a set of rewrite rules, that rewrite proofs in NL to proofs in
  \lamET.
\end{enumerate*}
With these two translation functions, we can set up a new presentation
for our syntactic proofs, namely one where the syntactic proofs are
labelled with the semantic terms resulting from the translation.

As an example, let's give an analysis of the sentence ``Mary likes
Bill.''. We assume the morphological, lexical and syntactic phases
have been taken care of. This turns our input sequent into:
\begin{center}
  \begin{pfbox}[0.9]
    \AXC{$\mary:\NP\prod\likes:\TV\prod\bill:\NP\;\fCenter\;?:\S$}
  \end{pfbox}
\end{center}%
Since our logic is propositional, proof search is decidable and we can
simply search the space of all possible proofs that end in this
sequent. We will find that the only proof is:
\begin{center}
  \hspace*{-0.25cm}%
  \begin{pfbox}[0.9]
    \AXC{}\RightLabel{Ax}\UIC{$\mary:\NP\fCenter\mary:\NP$}
    \AXC{}\RightLabel{Ax}\UIC{$\likes:\TV\fCenter\likes:\TV$}
    \RightLabel{$\impr$E}
    \BIC{$\mary:\NP\prod\likes:\TV\fCenter(\likes\;\mary):\S\impl\NP$}
    \AXC{}\RightLabel{Ax}\UIC{$\bill:\NP\fCenter\bill:\NP$}
    \RightLabel{$\impl$E}
    \BIC{$\mary:\NP\prod\likes:\TV\prod\bill:\NP\fCenter((\likes\;\mary)\;\bill):\S$}
  \end{pfbox}
\end{center}%
And so, by searching for a proof in our syntactic calculus, and
translating it to our semantic calculus, we get a meaning
representation for our input sequent.


\section{Display Calculus -- Research and Proof Search}\label{sec:display-calculus}
In the previous section, we glossed over the issue of proof
search. However, the natural deduction formulation of the syntactic
calculus in \autoref{fig:syntactic-calculus} is not especially suited
to proof search.
In this section, we will develop a display calculus
\citep{belnap1982} for NL based on work by
\citet{bernardi2010,bastenhof2012}\footnote{%
  While the cited work on focused proof search developed these
  techniques in order to obtain a CPS translation of the
  Lambek-Grishin calculus, we will adopt their techniques for a
  different reason: because they greatly reduce the search space, and
  therefore increase the efficiency of proof search. We will outfit
  our display calculus with a direct translation.
}. There are a few key advantages to this approach:
\begin{enumerate}
\item display calculi have a general cut-elimination theorem;
\item display calculi have the sub-formula property;
\end{enumerate}
The general cut-elimination theorem is incredibly useful when
developing a grammar logic, as any extension of the logic preserves
cut-elimination as long as it obeys the rules of display logic.
The sub-formula property allows the logic to be searched using naive
backward-chaining proof search. A further advantage of display
calculus is that it does not use contexts or plugging functions, as
used in \autoref{fig:explicit-semantic-calculus} and the usual sequent
calculus formulation of NL. This allows for the proof search to be
truly naive: we can say a rule applies if its conclusion can be
unified with the current proof obligation, and do not have to check
all possible contexts under which this unification could succeed.

\begin{figure}
  \begin{mdframed}
    \centering
    \begin{alignat*}{4}
      \text{Type}&\;        &A,B &\coloneqq \text{S}\vsep\text{N}\vsep\text{NP}\vsep\text{PP}\vsep A\impr B\vsep B\impl A\\
      \text{Structure}&^+\; &Γ   &\coloneqq \cdot A\cdot\vsep Γ_1\prod Γ_2\\
      \text{Structure}&^-\; &Δ   &\coloneqq \cdot A\cdot\vsep Γ\imprΔ\vsep Δ\implΓ\\
    \end{alignat*}

    \begin{pfbox}
      \AXC{} \RightLabel{Ax$^L$} \UIC{$[A]\fCenter A$}
    \end{pfbox}
    \begin{pfbox}
      \AXC{} \RightLabel{Ax$^R$} \UIC{$A\fCenter [A]$}
    \end{pfbox}

    \vspace*{\baselineskip}
    \begin{pfbox}
      \AXC{$[A]\fCenter Δ$}
      \doubleLine\RightLabel{Foc$^L$}
      \UIC{$\cdot A\cdot\fCenter Δ$}
    \end{pfbox}
    \begin{pfbox}
      \AXC{$Γ\fCenter[B]$}
      \doubleLine\RightLabel{Foc$^R$}
      \UIC{$Γ\fCenter\cdot B\cdot$}
    \end{pfbox}

    \vspace*{\baselineskip}
    \begin{pfbox}
      \AXC{$ Γ \fCenter[A]$}
      \AXC{$[B]\fCenter Δ $}
      \RightLabel{L$\impr$}
      \BIC{$[A\impr B]\fCenter Γ\impr Δ$}
    \end{pfbox}
    \begin{pfbox}
      \AXC{$Γ\fCenter\cdot A\cdot\impr\cdot B\cdot$}
      \RightLabel{R$\impr$}
      \UIC{$Γ\fCenter\cdot A\impr B\cdot$}
    \end{pfbox}

    \vspace*{\baselineskip}
    \begin{pfbox}
      \AXC{$ Γ \fCenter[A]$}
      \AXC{$[B]\fCenter Δ $}
      \RightLabel{L$\impl$}
      \BIC{$[B\impl A]\fCenter Δ\impl Γ$}
    \end{pfbox}
    \begin{pfbox}
      \AXC{$Γ\fCenter\cdot B\cdot\impl\cdot A\cdot$}
      \RightLabel{R$\impl$}
      \UIC{$Γ\fCenter\cdot B\impl A\cdot$}
    \end{pfbox}

    \vspace*{\baselineskip}
    \begin{pfbox}
      \AXC{$Γ_2\fCenter Γ_1\impr Δ$}
      \doubleLine\RightLabel{Res$\impr\prod$}
      \UIC{$Γ_1\prod Γ_2\fCenter Δ$}
      \doubleLine\RightLabel{Res$\impl\prod$}
      \UIC{$Γ_1\fCenter Δ\impl Γ_2$}
    \end{pfbox}
    \vspace*{\baselineskip}
  \end{mdframed}
  \caption{
    The syntactic calculus from~\autoref{fig:syntactic-calculus} as a
    display calculus.}%
  \label{fig:display-calculus}
\end{figure}

%%% Local Variables:
%%% mode: latex
%%% TeX-master: t
%%% End:

\begin{figure}
  \begin{mdframed}
    \centering
    \vspace*{1\baselineskip}
    \begin{minipage}{0.666\linewidth}
      \centering
      \(\text{Type}\;A,B\coloneqq\ldots\vsep A\& B\)
    \end{minipage}%
    \begin{minipage}{0.333\linewidth}
      \centering
      \(\text{Pol}(A \& B) \mapsto {+}\)
    \end{minipage}
    \\[1\baselineskip]
    \begin{pfbox}
      \AXC{$\focus{A}\fCenter Δ$}
      \RightLabel{L\&$_1$}
      \UIC{$\focus{A\& B}\fCenter Δ$}
    \end{pfbox}
    \begin{pfbox}
      \AXC{$\focus{B}\fCenter Δ$}
      \RightLabel{L\&$_2$}
      \UIC{$\focus{A\& B}\fCenter Δ$}
    \end{pfbox}
    \begin{pfbox}
      \AXC{$Γ\fCenter\focus{A}$}
      \AXC{$Γ\fCenter\focus{B}$}
      \RightLabel{R\&}
      \BIC{$Γ\fCenter\focus{A\& B}$}
    \end{pfbox}
    \\[1\baselineskip]
    \hrulefill
    \\[1\baselineskip]
    \(\tr[(A \& B)] \mapsto \tr[A]\times\tr[B]\)
    \setlength{\tabcolsep}{0pt}
    \hspace*{-0.5cm}%
    \begin{tabular}{c c c}
      \begin{pfbox}[0.9]
        \AXC{$\focus{A}\fCenter Δ$}
        \RightLabel{L\&$_1$}
        \UIC{$\focus{A\& B}\fCenter Δ$}
      \end{pfbox}
      &$\Longrightarrow$&
      \begin{pfbox}[0.9]
        \AXC{}\RightLabel{Ax}\UIC{$\tr[A]\times\tr[B]\fCenter\tr[A]\times\tr[B]$}
        \AXC{$\tr[A]\fCenter\tr[Δ]$}
        \RightLabel{Weak.}
        \UIC{$\tr[A]\prod\tr[B]\fCenter\tr[Δ]$}
        \RightLabel{$\times$E}
        \BIC{$\tr[A]\times\tr[B]\fCenter\tr[Δ]$}
      \end{pfbox}
      \\
      \begin{pfbox}[0.9]
        \AXC{$\focus{B}\fCenter Δ$} \RightLabel{L\&$_2$}
        \UIC{$\focus{A\& B}\fCenter Δ$}
      \end{pfbox}
      &$\Longrightarrow$&
      \begin{pfbox}[0.9]
        \AXC{}\RightLabel{Ax}\UIC{$\tr[A]\times\tr[B]\fCenter\tr[A]\times\tr[B]$}
        \AXC{$\tr[B]\fCenter\tr[Δ]$}
        \RightLabel{Weak.}
        \UIC{$\tr[A]\prod\tr[B]\fCenter\tr[Δ]$}
        \RightLabel{$\times$E}
        \BIC{$\tr[A]\times\tr[B]\fCenter\tr[Δ]$}
      \end{pfbox}
      \\
      \begin{pfbox}[0.9]
        \AXC{$Γ\fCenter\focus{A}$} \AXC{$Γ\fCenter\focus{B}$}
        \RightLabel{R\&} \BIC{$Γ\fCenter\focus{A\& B}$}
      \end{pfbox}
      &$\Longrightarrow$&
      \begin{pfbox}[0.9]
        \AXC{$\tr[Γ]\fCenter\tr[A]$}
        \AXC{$\tr[Γ]\fCenter\tr[B]$}
        \RightLabel{$\times$I}
        \BIC{$\tr[Γ]\prod\tr[Γ]\fCenter\tr[A]\times\tr[B]$}
        \RightLabel{Cont.}
        \UIC{$\tr[Γ]\fCenter\tr[A]\times\tr[B]$}
      \end{pfbox}
    \end{tabular}
    \vspace*{\baselineskip}
  \end{mdframed}
  \caption{
    Extension of calculus in \autoref{fig:display-calculus} which
    supports ambiguity.}%
  \label{fig:extension-lexical-ambiguity}
\end{figure}
%
%%% Local Variables:
%%% mode: latex
%%% TeX-master: t
%%% End:

\begin{figure}[hb]
  \begin{mdframed}
    \centering
    \begin{minipage}{0.666\linewidth}
      \centering
      \begin{alignat*}{4}
        \text{Type}     &  \;&A,B&\coloneqq\ldots\vsep A\himpr B\vsep B\himpl A\vsep\qr[A]\\
        \text{Structure}&^+\;&Γ  &\coloneqq\ldots\vsep Γ_1\hprod Γ_2\vsep\I\vsep\B\vsep\C\\
        \text{Structure}&^-\;&Δ  &\coloneqq\ldots\vsep Γ\himpr Δ\vsep Δ\himpl Γ
      \end{alignat*}
    \end{minipage}%
    \begin{minipage}{0.333\linewidth}
      \centering
      \begin{alignat*}{4}
        &\text{Pol}(A\himpr B) &&\mapsto{-}\\
        &\text{Pol}(B\himpl A) &&\mapsto{-}\\
        &\text{Pol}(\qr[A])    &&\mapsto{+}
      \end{alignat*}
    \end{minipage}
    \\[1\baselineskip]
    (copy of rules for $\{\impr,\prod,\impl\}$ from
    \autoref{fig:display-calculus} for $\{\himpr,\hprod,\himpl\}$)
    \\[1\baselineskip]
    \begin{pfbox}
      \AXC{$\struct{A}\hprod\I\fCenter Δ$}
      \RightLabel{L\I}
      \UIC{$\struct{\qr[A]}\fCenter Δ$}
    \end{pfbox}
    \begin{pfbox}
      \AXC{$Γ\fCenter\focus{B}$}
      \RightLabel{R\I}
      \UIC{$Γ\hprod\I\fCenter\focus{\qr[B]}$}
    \end{pfbox}
    \begin{pfbox}
      \AXC{$Γ\fCenter Δ$}
      \RightLabel{$\I^-$}
      \UIC{$Γ\hprod\I\fCenter Δ$}
    \end{pfbox}
    \\[1\baselineskip]
    \begin{pfbox}
      \AXC{$Γ_1\prod(Γ_2\hprod Γ_3)\fCenter Δ$}
      \doubleLine\RightLabel{\B}
      \UIC{$Γ_2\hprod((\B\prod Γ_1)\prod Γ_3)\fCenter Δ$}
    \end{pfbox}
    \begin{pfbox}
      \AXC{$(Γ_1\hprod Γ_2)\prod Γ_3\fCenter Δ$}
      \doubleLine\RightLabel{\C}
      \UIC{$Γ_1\hprod((\C\prod Γ_2)\prod Γ_3)\fCenter Δ$}
    \end{pfbox}
    \\[1\baselineskip]
    \hrulefill
    \\[1\baselineskip]
    {
      \renewcommand{\arraystretch}{1.5}%
      \(\!
      \begin{array}{c c c}
        \multicolumn{3}{c}{\qr[\tr[A]]\mapsto\tr[A]}\\
        \tr[\I]\mapsto\top      & \tr [\B]\mapsto\top     & \tr [\C]\mapsto\top\\
        \trd[\I]\mapsto\emptyset& \trd[\B]\mapsto\emptyset& \trd[\C]\mapsto\emptyset
      \end{array}
      \)
    }
    \\[1\baselineskip]
    (copy of translations for $\{\impr,\prod,\impl\}$ from
    \autoref{fig:display-calculus-to-explicit-lamET} for
    $\{\himpr,\hprod,\himpl\}$)\\
    \begin{tabular}{c c c}
      \begin{pfbox}
        \AXC{$\struct{A}\hprod\I\fCenter Δ$}
        \RightLabel{L\I}
        \UIC{$\struct{\qr[A]}\fCenter Δ$}
      \end{pfbox}
      &$\Longrightarrow$&
      \begin{pfbox}
        \AXC{$\tr[A]\prod\emptyset\fCenter\tr[Δ]$}
        \RightLabel{$\emptyset$E}
        \UIC{$\tr[A]\fCenter\tr[Δ]$}
      \end{pfbox}
      \\
      \begin{pfbox}
        \AXC{$Γ\fCenter\focus{B}$}
        \RightLabel{R\I}
        \UIC{$Γ\hprod\I\fCenter\focus{\qr[B]}$}
      \end{pfbox}
      &$\Longrightarrow$&
      \begin{pfbox}
        \AXC{$\trd[Γ]\fCenter\tr[B]$}
        \RightLabel{Weak.}
        \UIC{$\trd[Γ]\prod\emptyset\fCenter\tr[B]$}
      \end{pfbox}
      \\
      \begin{pfbox}
        \AXC{$Γ\fCenter Δ$}
        \RightLabel{$\I^-$}
        \UIC{$Γ\hprod\I\fCenter Δ$}
      \end{pfbox}
      &$\Longrightarrow$&
      \begin{pfbox}
        \AXC{$\trd[Γ]\fCenter\tr[Δ]$}
        \RightLabel{Weak.}
        \UIC{$\trd[Γ]\prod\emptyset\fCenter\tr[Δ]$}
      \end{pfbox}
    \end{tabular}
    \\
    \vspace*{\baselineskip}
    (\B\ and \C\ translate to various combinations of associativity,
    commutativity, $\emptyset$E and weakening)
    \\
    \vspace*{\baselineskip}
  \end{mdframed}
  \caption{
    Extension of calculus in \autoref{fig:display-calculus} which
    supports quantifier raising.}%
  \label{fig:extension-quantifier-raising}
\end{figure}

%%% Local Variables:
%%% mode: latex
%%% TeX-master: t
%%% End:

\begin{figure}
  \begin{mdframed}
    \centering
    \begin{alignat*}{4}
      \text{Type}     &  \;&A,B&\coloneqq \ldots\vsep\Diamond A\vsep\Box A\\
      \text{Structure}&^+\;&Γ  &\coloneqq \ldots\vsep\langle Γ\rangle\\
      \text{Structure}&^-\;&Δ  &\coloneqq \ldots\vsep[Δ]
    \end{alignat*}

    \begin{pfbox}
      \AXC{$\langle\struct{A}\rangle\fCenter Δ$}
      \RightLabel{L$\di$}
      \UIC{$\struct{\di A}\fCenter Δ$}
    \end{pfbox}
    \begin{pfbox}
      \AXC{$Γ\fCenter\focus{B}$}
      \RightLabel{R$\di$}
      \UIC{$\langle Γ\rangle\fCenter\focus{\di B}$}
    \end{pfbox}

    \vspace*{\baselineskip}
    \begin{pfbox}
      \AXC{$\focus{A}\fCenter Δ$}
      \RightLabel{L$\di$}
      \UIC{$\focus{\sq A}\fCenter[Δ]$}
    \end{pfbox}
    \begin{pfbox}
      \AXC{$Γ\fCenter[\struct{B}]$}
      \RightLabel{R$\sq$}
      \UIC{$Γ\fCenter\struct{\sq B}$}
    \end{pfbox}

    \vspace*{\baselineskip}
    \begin{pfbox}
      \AXC{$Γ\fCenter[Δ]$}
      \RightLabel{Res$\sq\di$}
      \UIC{$\langle Γ\rangle\fCenter Δ$}
    \end{pfbox}
    \vspace*{\baselineskip}
    \begin{pfbox}
      \AXC{$\langle Γ\rangle\fCenter Δ$}
      \RightLabel{Res$\di\sq$}
      \UIC{$Γ\fCenter[Δ]$}
    \end{pfbox}
    \vspace*{\baselineskip}
  \end{mdframed}
  \caption{
    Extension of calculus in \autoref{fig:extension-quantifier-raising}
    which supports scope islands.}%
  \label{fig:extension-scope-islands}
\end{figure}

%%% Local Variables:
%%% mode: latex
%%% TeX-master: t
%%% End:

\begin{figure}[hb]
  \begin{mdframed}
    \centering
    \begin{minipage}{0.66\linewidth}
      \begin{alignat*}{4}
        \text{Type}     &  \;&A,B&\coloneqq \ldots\vsep\diDn A\vsep\sqDn A\\
        \text{Structure}&^+\;&Γ  &\coloneqq \ldots\vsep\diDn Γ\\
        \text{Structure}&^-\;&Δ  &\coloneqq \ldots\vsep\sqDn Δ
      \end{alignat*}
    \end{minipage}%
    \begin{minipage}{0.33\linewidth}
      \begin{alignat*}{3}
        &A \downharpoonleft  B\;&&\coloneqq\;(\diDn\sqDn{A})\impr{B}\\
        &B \downharpoonright A\;&&\coloneqq\;{B}\impl(\diDn\sqDn{A})
      \end{alignat*}
    \end{minipage}
    \\
    \vspace*{\baselineskip}%
    (copy of rules for $\{\di,\sq\}$ from
    \autoref{fig:extension-scope-islands} for $\{\diDn,\sqDn\}$)
    \\
    \vspace*{\baselineskip}%
    \begin{pfbox}
      \AXC{$Γ_1\prod (Γ_2\prod\diDn Γ_3)\fCenter Δ$}
      \RightLabel{RR\diDn}
      \UIC{$(Γ_1\prod Γ_2)\prod\diDn Γ_3\fCenter Δ$}
    \end{pfbox}
    \begin{pfbox}
      \AXC{$(Γ_1\prod\diDn Γ_3)\prod Γ_2\fCenter Δ$}
      \RightLabel{LR\diDn}
      \UIC{$(Γ_1\prod Γ_2)\prod\diDn Γ_3\fCenter Δ$}
    \end{pfbox}

    \vspace*{\baselineskip}%
    \begin{pfbox}
      \AXC{$(\diDn Γ_3\prod Γ_2)\prod Γ_1\fCenter Δ$}
      \RightLabel{LL\diDn}
      \UIC{$\diDn Γ_3\prod (Γ_2\prod Γ_1)\fCenter Δ$}
    \end{pfbox}
    \begin{pfbox}
      \AXC{$Γ_2\prod (\diDn Γ_3\prod Γ_1)\fCenter Δ$}
      \RightLabel{RL\diDn}
      \UIC{$\diDn Γ_3\prod (Γ_2\prod Γ_1)\fCenter Δ$}
    \end{pfbox}
    \\
    \vspace*{\baselineskip}
    \hrulefill
    \\
    \vspace*{\baselineskip}
    (copy of translations for $\{\di,\sq\}$ from
    \autoref{fig:extension-scope-islands} for $\{\diDn,\sqDn\}$)
    \\
    \vspace*{\baselineskip}
    ({RR\diDn}, {LR\diDn}, {LL\diDn} and {RL\diDn} translate to
    various combinations of associativity and commutativity)
    \\
    \vspace*{\baselineskip}
  \end{mdframed}
  \caption{Extension of calculus in \autoref{fig:display-calculus}
    which supports infixation.}
  \label{fig:extension-infixation}
\end{figure}
%
%%% Local Variables:
%%% mode: latex
%%% TeX-master: t
%%% End:

\begin{figure}
  \begin{mdframed}
    \centering
    \begin{minipage}{0.66\linewidth}
      \begin{alignat*}{4}
        \text{Type}     &  \;&A,B&\coloneqq \ldots\vsep\diExt A\vsep\sqExt A\\
        \text{Structure}&^+\;&Γ  &\coloneqq \ldots\vsep\diExt Γ\\
        \text{Structure}&^-\;&Δ  &\coloneqq \ldots\vsep\sqExt Δ
      \end{alignat*}
    \end{minipage}%
    \begin{minipage}{0.33\linewidth}
      \begin{alignat*}{3}
        &A \downharpoonleft  B\;&&\coloneqq\;\diExt\sqExt A\impr B\\
        &B \downharpoonright A\;&&\coloneqq\;\diExt\sqExt B\impl A
      \end{alignat*}
    \end{minipage}
    \\
    \vspace*{\baselineskip}%
    (copy of rules for $\{\di,\sq\}$ from
    \autoref{fig:extension-scope-islands} for $\{\diExt,\sqExt\}$)
    \\
    \vspace*{\baselineskip}%
    \begin{pfbox}
      \AXC{$Γ_1\prod (Γ_2\prod\diExt Γ_3)\fCenter Δ$}
      \RightLabel{RR\diExt}
      \UIC{$(Γ_1\prod Γ_2)\prod\diExt Γ_3\fCenter Δ$}
    \end{pfbox}
    \begin{pfbox}
      \AXC{$(Γ_1\prod\diExt Γ_3)\prod Γ_2\fCenter Δ$}
      \RightLabel{LR\diExt}
      \UIC{$(Γ_1\prod Γ_2)\prod\diExt Γ_3\fCenter Δ$}
    \end{pfbox}

    \vspace*{\baselineskip}%
    \begin{pfbox}
      \AXC{$(\diExt Γ_3\prod Γ_2)\prod Γ_1\fCenter Δ$}
      \RightLabel{LL\diExt}
      \UIC{$\diExt Γ_3\prod (Γ_2\prod Γ_1)\fCenter Δ$}
    \end{pfbox}
    \begin{pfbox}
      \AXC{$Γ_2\prod (\diExt Γ_3\prod Γ_1)\fCenter Δ$}
      \RightLabel{RL\diExt}
      \UIC{$\diExt Γ_3\prod (Γ_2\prod Γ_1)\fCenter Δ$}
    \end{pfbox}
    \vspace*{\baselineskip}%
  \end{mdframed}
  \caption{Extension of calculus in \autoref{fig:display-calculus}
    which supports extraction.}
  \label{fig:extension-extraction}
\end{figure}
%
%%% Local Variables:
%%% mode: latex
%%% TeX-master: t
%%% End:


% - implicit semantic calculus;
% - explicit semantic calculus;
% - syntactic calculus;
% - display calculus;
% - compositionality principle;
% - problems with compositionality;
% - quantifier raising and scope ambiguity;
% - continuation monad & delimited continuations;
% - extension: lexical ambiguity;
% - extension: quantifier raising
%   * treatment of some & every;
%   * treatment of same & different;
%   * treatment of plurals;
% - extension: scope islands;

\section{Future work}

\paragraph*{Forward-Chaining Proof Search}
In \autoref{sec:what-is-type-logical-grammar} it was mentioned that
most research focuses on implementing what I call the `semantic
function' (i.e. interpreting). This is a good approach for research:
we can limit ourselves to sequent calculus, which has pleasant
properties, refer to the huge body of work on generative grammar to
inform our choice for sentence structure, and simply focus on making
these known structures derivable. However, in order to be feasible in
a practical system, one must also implement what I call the `syntactic
function' (i.e. parsing). One way of doing this is by switching to
forward-chaining proof search, i.e. by constructing all possible
sentences based on the given words, and filtering on those which are
both pronounceable and maintain the correct word-order.
\todo{Ask Michael for a reference.}

\begin{appendices}
  \begin{sidewaysfigure}
  \begin{mdframed}
  \begin{tabularx}{1.0\linewidth}{l c c r}
    \begin{pfbox}[0.7]
      \AXC{}\RightLabel{Ax}\UIC{$A\fCenter A$}
    \end{pfbox}
    &$\Longrightarrow$&
    \begin{pfbox}[0.7]
      \AXC{}\RightLabel{Ax}\UIC{$Π,A\fCenter A$}
      \RightLabel{$\ra$I}
      \UIC{$Π\fCenter A\ra A$}
    \end{pfbox}
    &
    \multirow{3}{*}{\fbox{\scriptsize\(\!
      \begin{aligned}
        &\tr[A]             &&\mapsto A                  \\
        &\tr[\emptyset]     &&\mapsto \top               \\
        &\tr[(Γ\prod Δ)]    &&\mapsto \tr[Γ]\times\tr[Δ] \\
        &\tr[(Γ\fCenter A)] &&\mapsto Π\fCenter\tr[Γ]\ra A
        \quad\text{\footnotesize(for arbitrary Π)}
      \end{aligned}
    \)}}\\

    \begin{pfbox}[0.7]
      \AXC{$Γ\prod A\fCenter B$}
      \RightLabel{$\ra${I}}
      \UIC{$Γ\fCenter A\ra B$}
    \end{pfbox}
    &$\Longrightarrow$&
    \begin{pfbox}[0.7]
      \AXC{$Π,\tr[Γ],A\fCenter\tr[Γ]\times A\ra B$}
      \AXC{}\RightLabel{Ax}\UIC{$Π,\tr[Γ],A\fCenter\tr[Γ]$}
      \AXC{}\RightLabel{Ax}\UIC{$Π,\tr[Γ],A\fCenter A$}
      \RightLabel{$\times$I}
      \BIC{$Π,\tr[Γ],A\fCenter\tr[Γ]\times A$}
      \RightLabel {$\ra$E}
      \BIC{$Π,\tr[Γ],A\fCenter B$}
      \RightLabel{$\ra$I}
      \UIC{$Π,\tr[Γ]\fCenter A\ra B$}
      \RightLabel{$\ra$I}
      \UIC{$Π\fCenter\tr[Γ]\ra A\ra B$}
    \end{pfbox}
    \\

    \begin{pfbox}[0.7]
      \AXC{$Γ\fCenter A\ra B$}
      \AXC{$Δ\fCenter A$}
      \RightLabel{$\ra${E}}
      \BIC{$Γ\prod Δ\fCenter B$}
    \end{pfbox}
    &$\Longrightarrow$&
    \\
    \multicolumn{4}{r}{%
    \begin{pfbox}[0.7]
      \AXC{}\RightLabel{Ax}\UIC{$Π,\tr[Γ]\times\tr[Δ]\fCenter\tr[Γ]\times\tr[Δ]$}
      \AXC{$Π,\tr[Γ]\times\tr[Δ],\tr[Γ],\tr[Δ]\fCenter \tr[Γ]\ra A\ra B$}
      \AXC{}\RightLabel{Ax}\UIC{$Π,\tr[Γ]\times\tr[Δ],\tr[Γ],\tr[Δ]\fCenter\tr[Γ]$}
      \RightLabel{$\ra$E}
      \BIC{$Π,\tr[Γ]\times\tr[Δ],\tr[Γ],\tr[Δ]\fCenter A\ra B$}
      \AXC{$Π,\tr[Γ]\times\tr[Δ],\tr[Γ],\tr[Δ]\fCenter \tr[Δ]\ra A$}
      \AXC{}\RightLabel{Ax}\UIC{$Π,\tr[Γ]\times\tr[Δ],\tr[Γ],\tr[Δ]\fCenter \tr[Δ]$}
      \RightLabel{$\ra$E}
      \BIC{$Π,\tr[Γ]\times\tr[Δ],\tr[Γ],\tr[Δ]\fCenter A$}
      \RightLabel{$\ra$E}
      \BIC{$Π,\tr[Γ]\times\tr[Δ],\tr[Γ],\tr[Δ]\fCenter B$}
      \RightLabel{$\times$E}
      \BIC{$Π,\tr[Γ]\times\tr[Δ]\fCenter B$}
      \RightLabel{$\ra$I}
      \UIC{$Π\fCenter\tr[Γ]\times\tr[Δ]\ra B$}
    \end{pfbox}
    }
    \\

    \begin{pfbox}[0.7]
      \AXC{$Γ\fCenter A$}
      \AXC{$Δ\fCenter B$}
      \RightLabel{$\times$I}
      \BIC{$Γ\prod Δ\fCenter A\times B$}
    \end{pfbox}
    &$\Longrightarrow$&
    \\
    \multicolumn{4}{r}{%
    \begin{pfbox}[0.7]
      \AXC{}\RightLabel{Ax}\UIC{$Π,\tr[Γ]\times\tr[Δ]\fCenter\tr[Γ]\times\tr[Δ]$}
      \AXC{$Π,\tr[Γ]\times\tr[Δ],\tr[Γ],\tr[Δ]\fCenter\tr[Γ]\ra A$}
      \AXC{}\RightLabel{Ax}\UIC{$Π,\tr[Γ]\times\tr[Δ],\tr[Γ],\tr[Δ]\fCenter\tr[Γ]$}
      \RightLabel{$\ra$E}
      \BIC{$Π,\tr[Γ]\times\tr[Δ],\tr[Γ],\tr[Δ]\fCenter A$}
      \AXC{$Π,\tr[Γ]\times\tr[Δ],\tr[Γ],\tr[Δ]\fCenter\tr[Δ]\ra B$}
      \AXC{}\RightLabel{Ax}\UIC{$Π,\tr[Γ]\times\tr[Δ],\tr[Γ],\tr[Δ]\fCenter\tr[Δ]$}
      \RightLabel{$\ra$E}
      \BIC{$Π,\tr[Γ]\times\tr[Δ],\tr[Γ],\tr[Δ]\fCenter B$}
      \RightLabel{$\times$I}
      \BIC{$Π,\tr[Γ]\times\tr[Δ],\tr[Γ],\tr[Δ]\fCenter A\times B$}
      \RightLabel{$\times$E}
      \BIC{$Π,\tr[Γ]\times\tr[Δ]\fCenter A\times B$}
      \RightLabel{$\ra$I}
      \UIC{$Π\fCenter\tr[Γ]\times\tr[Δ]\ra A\times B$}
    \end{pfbox}
    }
    \\

    \begin{pfbox}[0.7]
      \AXC{$Γ\fCenter A\times B$}
      \AXC{$Δ\prod A\prod B\fCenter C$}
      \RightLabel{$\times$E}
      \BIC{$Γ\prod Δ\fCenter C$}
    \end{pfbox}
    &$\Longrightarrow$&
    \\
    \multicolumn{4}{r}{%
    \begin{pfbox}[0.7]
      \AXC{}\RightLabel{Ax}\UIC{$Π,\tr[Γ]\times\tr[Δ]\fCenter\tr[Γ]\times\tr[Δ]$}
      \AXC{$Π,\tr[Γ]\times\tr[Δ],\tr[Γ],\tr[Δ]\fCenter\tr[Γ]\ra A\times B$}
      \AXC{}\RightLabel{Ax}\UIC{$Π,\tr[Γ]\times\tr[Δ],\tr[Γ],\tr[Δ]\fCenter\tr[Γ]$}
      \RightLabel{$\ra$E}
      \BIC{$Π,\tr[Γ]\times\tr[Δ],\tr[Γ],\tr[Δ]\fCenter A\times B$}
      \AXC{$Π,\tr[Γ]\times\tr[Δ],\tr[Γ],\tr[Δ],A,B\fCenter\tr[Δ]\ra C$}
      \AXC{}\RightLabel{Ax}\UIC{$Π,\tr[Γ]\times\tr[Δ],\tr[Γ],\tr[Δ],A,B\fCenter\tr[Δ]$}
      \RightLabel{$\ra$E}
      \BIC{$Π,\tr[Γ]\times\tr[Δ],\tr[Γ],\tr[Δ],A,B\fCenter C$}
      \RightLabel{$\times$E}
      \BIC{$Π,\tr[Γ]\times\tr[Δ],\tr[Γ],\tr[Δ]\fCenter C$}
      \RightLabel{$\times$E}
      \BIC{$Π,\tr[Γ]\times\tr[Δ]\fCenter C$}
      \RightLabel{$\ra$I}
      \UIC{$Π\fCenter\tr[Γ]\times\tr[Δ]\ra C$}
    \end{pfbox}
    }
  \end{tabularx}%
  \vspace*{\baselineskip}
  \end{mdframed}%
  \caption{Translation from explicit to implicit version of \lamET\ as
    used in the Haskell implementation.}%
  \label{fig:explicit-to-implicit}
\end{sidewaysfigure}%
%
%%% Local Variables:
%%% mode: latex
%%% TeX-master: t
%%% End:

\end{appendices}

\bibliographystyle{apalike}%
\bibliography{main}%

\end{document}
