\documentclass[a4paper]{article}

\usepackage[usenames,dvipsnames]{color}
\usepackage[
  colorlinks = true,
  linkcolor  = Maroon,
  urlcolor   = MidnightBlue,
  citecolor  = ForestGreen
  ]{hyperref}
\usepackage{natbib}
\usepackage{ifthen}
\renewcommand{\hscodestyle}{\small}


% use Wouter's setup for notes (extended with ``hidden'')
\newboolean{marginNotes}
\setboolean{marginNotes}{false}

\newboolean{showNotes}
\setboolean{showNotes}{true}

\newboolean{showHidden}
\setboolean{showHidden}{false}

\newcommand{\hidden}[1]{
  \ifthenelse{\boolean{showHidden}}
             {#1}
             {}}
\newcommand{\marginNote}[1]{
  \ifthenelse{\boolean{marginNotes}}
             {\marginpar{#1}}
             {#1}}
\newcommand{\todo}[1]{
  \ifthenelse{\boolean{showNotes}}
             {\marginNote{\textcolor{red}{\textbf{TODO:~}#1}}}
             {}}

\renewcommand{\sectionautorefname}{\S}
\renewcommand{\subsectionautorefname}{\S\S}

% use Buss' Proofs
\usepackage{mdframed}
\usepackage{bussproofs}
\EnableBpAbbreviations%
\def\fCenter{\ \vdash\ }
\def\limpl{\multimap}
\newenvironment{scprooftree}[1]%
  {\gdef\scalefactor{#1}\begin{center}\proofSkipAmount \leavevmode}%
  {\scalebox{\scalefactor}{\DisplayProof}\proofSkipAmount \end{center} }

% define various which I'll be sure to use symbols
\usepackage{textgreek}
\usepackage{stmaryrd}
\usepackage{relsize}
\usepackage{ucs}
\usepackage[utf8x]{inputenc}
\DeclareUnicodeCharacter{8656}{$\slash$}
\DeclareUnicodeCharacter{8658}{$\backslash$}
\DeclareUnicodeCharacter{8666}{$\varoslash$}
\DeclareUnicodeCharacter{8667}{$\varobslash$}
\DeclareUnicodeCharacter{7478}{$^{J}$}
\DeclareUnicodeCharacter{7480}{$^{L}$}
\DeclareUnicodeCharacter{7486}{$^{P}$}
\DeclareUnicodeCharacter{7487}{$^{R}$}


% customise the appearance of itemise bullets
\def\myitem{\item[$\circ$]}

\newtheorem{lemma}{Lemma}

% 1. simple semantic calculus;
% 2. simple syntactic calculus;
% 3. compositionality principle;
% 4. problems with compositionality;
% 5. extended semantic calculus;
% 6. quantifier raising and scope ambiguity;
% 7. continuation monad;
% 8. delimited continuations and indexed monads;
% 9. extended syntactic calculus;

\begin{document}

%\begin{figure}
  \centering
  \begin{minipage}{0.35\linewidth}%
    \centering
    \framebox{Morphological}\\
    $\downarrow$\\
    \framebox{Lexical}\\
    $\downarrow$\\
    \framebox{Syntactic}\\
    $\downarrow$\\
    \framebox{Semantic}\\
    $\downarrow$\\
    \framebox{Pragmatic}\\
  \end{minipage}
  \begin{minipage}{0.55\linewidth}%
    \centering
    ``Mary saw foxes.''\\
    $\downarrow$\\
    Mary see.PAST fox.PL\\
    $\downarrow$\\
    Mary:NP see:TV.PAST fox:NP.PL\\
    $\downarrow$\\
    Mary:NP [see:TV.PAST fox:NP.PL]\\
    $\downarrow$\\
    $\exists X. X \subseteq\mathbf{fox}\land\mathbf{past}(\mathbf{see}(\text{Mary},X))$\\
    $\downarrow$\\
    \ldots\\
  \end{minipage}
  \caption{An abstract pipeline for natural language understanding.}%
  \label{fig:nlu-pipeline}
\end{figure}

%%% Local Variables:
%%% mode: latex
%%% TeX-master: t
%%% End:

%\begin{figure}
  \begin{mdframed}
    \centering
    \begin{alignat*}{4}
      \text{Type}       \;&A,B&&\coloneqq \e\vsep\t\vsep A\ra B\\
      \text{Term}       \;&M,N&&\coloneqq x\vsep C\vsep\lambda x.M\vsep(M\;N)\\
      \text{Constant}   \;&C  &&\coloneqq
      {\forall}\vsep{\exists}\vsep{\neg}\vsep{\supset}\vsep{\land}\vsep{\lor}\vsep\ldots\\
      \text{Environment}\;&Γ  &&\text{set of typing assumptions of the form `$M : A$'}
    \end{alignat*}

    \begin{pfbox}
      \AXC{$(x : A)\in Γ$} \RightLabel{{Ax}} \UIC{$Γ\fCenter x : A$}
    \end{pfbox}

    \vspace*{\baselineskip}
    \begin{pfbox}
      \AXC{$Γ,x : A\fCenter M : B$} \RightLabel{$\ra${I}}
      \UIC{$Γ\fCenter \lambda x.M : A\ra B$}
    \end{pfbox}
    \begin{pfbox}
      \AXC{$Γ\fCenter M : A\ra B$} \AXC{$Γ\fCenter N : A$}
      \RightLabel{$\ra${E}} \BIC{$Γ\fCenter (M\;N) : B$}
    \end{pfbox}

    \vspace*{\baselineskip}
  \end{mdframed}
  \caption{A simple semantic calculus.}%
  \label{fig:implicit-semantic-calculus}
\end{figure}
%
%%% Local Variables:
%%% mode: latex
%%% TeX-master: t
%%% End:

%\begin{figure}
  \begin{mdframed}
    \centering
    \begin{minipage}{0.6\linewidth}
      \begin{alignat*}{4}
        \text{Atom}     \;&α    &&\coloneqq \e\vsep\t\\
        \text{Type}     \;&A,B  &&\coloneqq α\vsep A\ra B\\
        \text{Structure}\;&Γ,Δ,Π&&\coloneqq A\vsep\emptyset\vsep Γ\prod Δ\\
        \text{Context}  \;&Σ    &&\coloneqq \Box\vsep Σ\prodl Δ\vsep Γ\prodr Σ
      \end{alignat*}
    \end{minipage}%
    \begin{minipage}{0.4\linewidth}
      \begin{align*}
        \Box [Γ]&\mapsto Γ\\
        (Σ\prodl Δ)[Γ]&\mapsto (Σ[Γ]\prod Δ)\\
        (Δ\prodr Σ)[Γ]&\mapsto (Δ\prod Σ[Γ])
      \end{align*}
    \end{minipage}

    \vspace*{\baselineskip}
    \begin{pfbox}[0.9]
      \AXC{}\RightLabel{{Ax}}\UIC{$A\fCenter A$}
    \end{pfbox}

    \vspace*{\baselineskip}
    \begin{pfbox}[0.9]
      \AXC{$Γ\prod A\fCenter B$}
      \RightLabel{$\ra${I}}
      \UIC{$Γ\fCenter A\ra B$}
    \end{pfbox}
    \begin{pfbox}[0.9]
      \AXC{$Γ\fCenter A\ra B$}
      \AXC{$Δ\fCenter A$}
      \RightLabel{$\ra${E}}
      \BIC{$Γ\prod Δ\fCenter B$}
    \end{pfbox}

    \vspace*{\baselineskip}
    \begin{pfbox}[0.9]
      \AXC{$Σ[Γ\prod \emptyset]\fCenter B$}
      \RightLabel{$\emptyset${E}}
      \UIC{$Σ[Γ]\fCenter B$}
    \end{pfbox}

    \vspace*{\baselineskip}
    \begin{pfbox}[0.9]
      \AXC{$Σ[A\prod A]\fCenter B$}
      \RightLabel{Cont.}
      \UIC{$Σ[A]\fCenter B$}
    \end{pfbox}
    \begin{pfbox}[0.9]
      \AXC{$Σ[Γ]\fCenter B$}
      \RightLabel{Weak.}
      \UIC{$Σ[Γ\prod Δ]\fCenter B$}
    \end{pfbox}

    \vspace*{\baselineskip}
    \begin{pfbox}[0.9]
      \AXC{$Σ[Δ\prod Γ]\fCenter B$}
      \RightLabel{Comm.}
      \UIC{$Σ[Γ\prod Δ]\fCenter B$}
    \end{pfbox}
    \begin{pfbox}[0.9]
      \AXC{$Σ[(Γ\prod Δ)\prod Π]\fCenter B$}
      \doubleLine\RightLabel{Ass.}
      \UIC{$Σ[Γ\prod (Δ\prod Π)]\fCenter B$}
    \end{pfbox}
    \vspace*{\baselineskip}
  \end{mdframed}
  \caption{Explicit \lamET, a simple semantic calculus with explicit
    structural rules.}%
  \label{fig:explicit-semantic-calculus}
\end{figure}
%
%%% Local Variables:
%%% mode: latex
%%% TeX-master: t
%%% End:

%\begin{figure}
  \begin{mdframed}
    \centering
    \begin{alignat*}{4}
      \text{Type}     \;&A,B&&\coloneqq\text{S}\vsep\text{N}\vsep\text{NP}\vsep\text{PP}\vsep \text{INF}\vsep A\impr B\vsep B\impl A\\
      \text{Structure}\;&Γ,Δ&&\coloneqq A\vsep Γ\prodΔ\\
    \end{alignat*}

    \begin{pfbox}
      \AXC{} \RightLabel{{Ax}} \UIC{$A\fCenter A$}
    \end{pfbox}

    \vspace*{\baselineskip}
    \begin{pfbox}
      \AXC{$A\prodΓ\fCenter B$} \RightLabel{$\impr${I}}
      \UIC{$Γ\fCenter A\impr B$}
    \end{pfbox}
    \begin{pfbox}
      \AXC{$Γ\fCenter A$} \AXC{$Δ\fCenter A\impr B$} \RightLabel{$\impr${E}}
      \BIC{$Γ\prodΔ\fCenter B$}
    \end{pfbox}

    \vspace*{\baselineskip}
    \begin{pfbox}
      \AXC{$Γ\prod A\fCenter B$} \RightLabel{$\impl${I}}
      \UIC{$Γ\fCenter B\impl A$}
    \end{pfbox}
    \begin{pfbox}
      \AXC{$Γ\fCenter B\impl A$} \AXC{$Δ\fCenter A$} \RightLabel{$\impl${E}}
      \BIC{$Γ\prod Δ\fCenter B$}
    \end{pfbox}

    \vspace*{\baselineskip}
  \end{mdframed}
  \caption{A simple syntactic calculus.}%
  \label{fig:syntactic-calculus}
\end{figure}
%
%%% Local Variables:
%%% mode: latex
%%% TeX-master: t
%%% End:

\begin{figure}
  \begin{mdframed}
    \centering
    \begin{alignat*}{4}
      \text{Type}&\;        &A,B &\coloneqq \text{S}\vsep\text{N}\vsep\text{NP}\vsep\text{PP}\vsep A\impr B\vsep B\impl A\\
      \text{Structure}&^+\; &Γ   &\coloneqq \cdot A\cdot\vsep Γ_1\prod Γ_2\\
      \text{Structure}&^-\; &Δ   &\coloneqq \cdot A\cdot\vsep Γ\imprΔ\vsep Δ\implΓ\\
    \end{alignat*}

    \begin{pfbox}
      \AXC{} \RightLabel{Ax$^L$} \UIC{$[A]\fCenter A$}
    \end{pfbox}
    \begin{pfbox}
      \AXC{} \RightLabel{Ax$^R$} \UIC{$A\fCenter [A]$}
    \end{pfbox}

    \vspace*{\baselineskip}
    \begin{pfbox}
      \AXC{$[A]\fCenter Δ$}
      \doubleLine\RightLabel{Foc$^L$}
      \UIC{$\cdot A\cdot\fCenter Δ$}
    \end{pfbox}
    \begin{pfbox}
      \AXC{$Γ\fCenter[B]$}
      \doubleLine\RightLabel{Foc$^R$}
      \UIC{$Γ\fCenter\cdot B\cdot$}
    \end{pfbox}

    \vspace*{\baselineskip}
    \begin{pfbox}
      \AXC{$ Γ \fCenter[A]$}
      \AXC{$[B]\fCenter Δ $}
      \RightLabel{L$\impr$}
      \BIC{$[A\impr B]\fCenter Γ\impr Δ$}
    \end{pfbox}
    \begin{pfbox}
      \AXC{$Γ\fCenter\cdot A\cdot\impr\cdot B\cdot$}
      \RightLabel{R$\impr$}
      \UIC{$Γ\fCenter\cdot A\impr B\cdot$}
    \end{pfbox}

    \vspace*{\baselineskip}
    \begin{pfbox}
      \AXC{$ Γ \fCenter[A]$}
      \AXC{$[B]\fCenter Δ $}
      \RightLabel{L$\impl$}
      \BIC{$[B\impl A]\fCenter Δ\impl Γ$}
    \end{pfbox}
    \begin{pfbox}
      \AXC{$Γ\fCenter\cdot B\cdot\impl\cdot A\cdot$}
      \RightLabel{R$\impl$}
      \UIC{$Γ\fCenter\cdot B\impl A\cdot$}
    \end{pfbox}

    \vspace*{\baselineskip}
    \begin{pfbox}
      \AXC{$Γ_2\fCenter Γ_1\impr Δ$}
      \doubleLine\RightLabel{Res$\impr\prod$}
      \UIC{$Γ_1\prod Γ_2\fCenter Δ$}
      \doubleLine\RightLabel{Res$\impl\prod$}
      \UIC{$Γ_1\fCenter Δ\impl Γ_2$}
    \end{pfbox}
    \vspace*{\baselineskip}
  \end{mdframed}
  \caption{
    The syntactic calculus from~\autoref{fig:syntactic-calculus} as a
    display calculus.}%
  \label{fig:display-calculus}
\end{figure}

%%% Local Variables:
%%% mode: latex
%%% TeX-master: t
%%% End:

%\begin{figure}
  \begin{mdframed}
    \centering
    \vspace*{1\baselineskip}
    \begin{minipage}{0.666\linewidth}
      \centering
      \(\text{Type}\;A,B\coloneqq\ldots\vsep A\& B\)
    \end{minipage}%
    \begin{minipage}{0.333\linewidth}
      \centering
      \(\text{Pol}(A \& B) \mapsto {+}\)
    \end{minipage}
    \\[1\baselineskip]
    \begin{pfbox}
      \AXC{$\focus{A}\fCenter Δ$}
      \RightLabel{L\&$_1$}
      \UIC{$\focus{A\& B}\fCenter Δ$}
    \end{pfbox}
    \begin{pfbox}
      \AXC{$\focus{B}\fCenter Δ$}
      \RightLabel{L\&$_2$}
      \UIC{$\focus{A\& B}\fCenter Δ$}
    \end{pfbox}
    \begin{pfbox}
      \AXC{$Γ\fCenter\focus{A}$}
      \AXC{$Γ\fCenter\focus{B}$}
      \RightLabel{R\&}
      \BIC{$Γ\fCenter\focus{A\& B}$}
    \end{pfbox}
    \\[1\baselineskip]
    \hrulefill
    \\[1\baselineskip]
    \(\tr[(A \& B)] \mapsto \tr[A]\times\tr[B]\)
    \setlength{\tabcolsep}{0pt}
    \hspace*{-0.5cm}%
    \begin{tabular}{c c c}
      \begin{pfbox}[0.9]
        \AXC{$\focus{A}\fCenter Δ$}
        \RightLabel{L\&$_1$}
        \UIC{$\focus{A\& B}\fCenter Δ$}
      \end{pfbox}
      &$\Longrightarrow$&
      \begin{pfbox}[0.9]
        \AXC{}\RightLabel{Ax}\UIC{$\tr[A]\times\tr[B]\fCenter\tr[A]\times\tr[B]$}
        \AXC{$\tr[A]\fCenter\tr[Δ]$}
        \RightLabel{Weak.}
        \UIC{$\tr[A]\prod\tr[B]\fCenter\tr[Δ]$}
        \RightLabel{$\times$E}
        \BIC{$\tr[A]\times\tr[B]\fCenter\tr[Δ]$}
      \end{pfbox}
      \\
      \begin{pfbox}[0.9]
        \AXC{$\focus{B}\fCenter Δ$} \RightLabel{L\&$_2$}
        \UIC{$\focus{A\& B}\fCenter Δ$}
      \end{pfbox}
      &$\Longrightarrow$&
      \begin{pfbox}[0.9]
        \AXC{}\RightLabel{Ax}\UIC{$\tr[A]\times\tr[B]\fCenter\tr[A]\times\tr[B]$}
        \AXC{$\tr[B]\fCenter\tr[Δ]$}
        \RightLabel{Weak.}
        \UIC{$\tr[A]\prod\tr[B]\fCenter\tr[Δ]$}
        \RightLabel{$\times$E}
        \BIC{$\tr[A]\times\tr[B]\fCenter\tr[Δ]$}
      \end{pfbox}
      \\
      \begin{pfbox}[0.9]
        \AXC{$Γ\fCenter\focus{A}$} \AXC{$Γ\fCenter\focus{B}$}
        \RightLabel{R\&} \BIC{$Γ\fCenter\focus{A\& B}$}
      \end{pfbox}
      &$\Longrightarrow$&
      \begin{pfbox}[0.9]
        \AXC{$\tr[Γ]\fCenter\tr[A]$}
        \AXC{$\tr[Γ]\fCenter\tr[B]$}
        \RightLabel{$\times$I}
        \BIC{$\tr[Γ]\prod\tr[Γ]\fCenter\tr[A]\times\tr[B]$}
        \RightLabel{Cont.}
        \UIC{$\tr[Γ]\fCenter\tr[A]\times\tr[B]$}
      \end{pfbox}
    \end{tabular}
    \vspace*{\baselineskip}
  \end{mdframed}
  \caption{
    Extension of calculus in \autoref{fig:display-calculus} which
    supports ambiguity.}%
  \label{fig:extension-lexical-ambiguity}
\end{figure}
%
%%% Local Variables:
%%% mode: latex
%%% TeX-master: t
%%% End:

\begin{figure}[hb]
  \begin{mdframed}
    \centering
    \begin{minipage}{0.666\linewidth}
      \centering
      \begin{alignat*}{4}
        \text{Type}     &  \;&A,B&\coloneqq\ldots\vsep A\himpr B\vsep B\himpl A\vsep\qr[A]\\
        \text{Structure}&^+\;&Γ  &\coloneqq\ldots\vsep Γ_1\hprod Γ_2\vsep\I\vsep\B\vsep\C\\
        \text{Structure}&^-\;&Δ  &\coloneqq\ldots\vsep Γ\himpr Δ\vsep Δ\himpl Γ
      \end{alignat*}
    \end{minipage}%
    \begin{minipage}{0.333\linewidth}
      \centering
      \begin{alignat*}{4}
        &\text{Pol}(A\himpr B) &&\mapsto{-}\\
        &\text{Pol}(B\himpl A) &&\mapsto{-}\\
        &\text{Pol}(\qr[A])    &&\mapsto{+}
      \end{alignat*}
    \end{minipage}
    \\[1\baselineskip]
    (copy of rules for $\{\impr,\prod,\impl\}$ from
    \autoref{fig:display-calculus} for $\{\himpr,\hprod,\himpl\}$)
    \\[1\baselineskip]
    \begin{pfbox}
      \AXC{$\struct{A}\hprod\I\fCenter Δ$}
      \RightLabel{L\I}
      \UIC{$\struct{\qr[A]}\fCenter Δ$}
    \end{pfbox}
    \begin{pfbox}
      \AXC{$Γ\fCenter\focus{B}$}
      \RightLabel{R\I}
      \UIC{$Γ\hprod\I\fCenter\focus{\qr[B]}$}
    \end{pfbox}
    \begin{pfbox}
      \AXC{$Γ\fCenter Δ$}
      \RightLabel{$\I^-$}
      \UIC{$Γ\hprod\I\fCenter Δ$}
    \end{pfbox}
    \\[1\baselineskip]
    \begin{pfbox}
      \AXC{$Γ_1\prod(Γ_2\hprod Γ_3)\fCenter Δ$}
      \doubleLine\RightLabel{\B}
      \UIC{$Γ_2\hprod((\B\prod Γ_1)\prod Γ_3)\fCenter Δ$}
    \end{pfbox}
    \begin{pfbox}
      \AXC{$(Γ_1\hprod Γ_2)\prod Γ_3\fCenter Δ$}
      \doubleLine\RightLabel{\C}
      \UIC{$Γ_1\hprod((\C\prod Γ_2)\prod Γ_3)\fCenter Δ$}
    \end{pfbox}
    \\[1\baselineskip]
    \hrulefill
    \\[1\baselineskip]
    {
      \renewcommand{\arraystretch}{1.5}%
      \(\!
      \begin{array}{c c c}
        \multicolumn{3}{c}{\qr[\tr[A]]\mapsto\tr[A]}\\
        \tr[\I]\mapsto\top      & \tr [\B]\mapsto\top     & \tr [\C]\mapsto\top\\
        \trd[\I]\mapsto\emptyset& \trd[\B]\mapsto\emptyset& \trd[\C]\mapsto\emptyset
      \end{array}
      \)
    }
    \\[1\baselineskip]
    (copy of translations for $\{\impr,\prod,\impl\}$ from
    \autoref{fig:display-calculus-to-explicit-lamET} for
    $\{\himpr,\hprod,\himpl\}$)\\
    \begin{tabular}{c c c}
      \begin{pfbox}
        \AXC{$\struct{A}\hprod\I\fCenter Δ$}
        \RightLabel{L\I}
        \UIC{$\struct{\qr[A]}\fCenter Δ$}
      \end{pfbox}
      &$\Longrightarrow$&
      \begin{pfbox}
        \AXC{$\tr[A]\prod\emptyset\fCenter\tr[Δ]$}
        \RightLabel{$\emptyset$E}
        \UIC{$\tr[A]\fCenter\tr[Δ]$}
      \end{pfbox}
      \\
      \begin{pfbox}
        \AXC{$Γ\fCenter\focus{B}$}
        \RightLabel{R\I}
        \UIC{$Γ\hprod\I\fCenter\focus{\qr[B]}$}
      \end{pfbox}
      &$\Longrightarrow$&
      \begin{pfbox}
        \AXC{$\trd[Γ]\fCenter\tr[B]$}
        \RightLabel{Weak.}
        \UIC{$\trd[Γ]\prod\emptyset\fCenter\tr[B]$}
      \end{pfbox}
      \\
      \begin{pfbox}
        \AXC{$Γ\fCenter Δ$}
        \RightLabel{$\I^-$}
        \UIC{$Γ\hprod\I\fCenter Δ$}
      \end{pfbox}
      &$\Longrightarrow$&
      \begin{pfbox}
        \AXC{$\trd[Γ]\fCenter\tr[Δ]$}
        \RightLabel{Weak.}
        \UIC{$\trd[Γ]\prod\emptyset\fCenter\tr[Δ]$}
      \end{pfbox}
    \end{tabular}
    \\
    \vspace*{\baselineskip}
    (\B\ and \C\ translate to various combinations of associativity,
    commutativity, $\emptyset$E and weakening)
    \\
    \vspace*{\baselineskip}
  \end{mdframed}
  \caption{
    Extension of calculus in \autoref{fig:display-calculus} which
    supports quantifier raising.}%
  \label{fig:extension-quantifier-raising}
\end{figure}

%%% Local Variables:
%%% mode: latex
%%% TeX-master: t
%%% End:

%\begin{figure}
  \begin{mdframed}
    \centering
    \begin{alignat*}{4}
      \text{Type}     &  \;&A,B&\coloneqq \ldots\vsep\Diamond A\vsep\Box A\\
      \text{Structure}&^+\;&Γ  &\coloneqq \ldots\vsep\langle Γ\rangle\\
      \text{Structure}&^-\;&Δ  &\coloneqq \ldots\vsep[Δ]
    \end{alignat*}

    \begin{pfbox}
      \AXC{$\langle\struct{A}\rangle\fCenter Δ$}
      \RightLabel{L$\di$}
      \UIC{$\struct{\di A}\fCenter Δ$}
    \end{pfbox}
    \begin{pfbox}
      \AXC{$Γ\fCenter\focus{B}$}
      \RightLabel{R$\di$}
      \UIC{$\langle Γ\rangle\fCenter\focus{\di B}$}
    \end{pfbox}

    \vspace*{\baselineskip}
    \begin{pfbox}
      \AXC{$\focus{A}\fCenter Δ$}
      \RightLabel{L$\di$}
      \UIC{$\focus{\sq A}\fCenter[Δ]$}
    \end{pfbox}
    \begin{pfbox}
      \AXC{$Γ\fCenter[\struct{B}]$}
      \RightLabel{R$\sq$}
      \UIC{$Γ\fCenter\struct{\sq B}$}
    \end{pfbox}

    \vspace*{\baselineskip}
    \begin{pfbox}
      \AXC{$Γ\fCenter[Δ]$}
      \RightLabel{Res$\sq\di$}
      \UIC{$\langle Γ\rangle\fCenter Δ$}
    \end{pfbox}
    \vspace*{\baselineskip}
    \begin{pfbox}
      \AXC{$\langle Γ\rangle\fCenter Δ$}
      \RightLabel{Res$\di\sq$}
      \UIC{$Γ\fCenter[Δ]$}
    \end{pfbox}
    \vspace*{\baselineskip}
  \end{mdframed}
  \caption{
    Extension of calculus in \autoref{fig:extension-quantifier-raising}
    which supports scope islands.}%
  \label{fig:extension-scope-islands}
\end{figure}

%%% Local Variables:
%%% mode: latex
%%% TeX-master: t
%%% End:



\subsubsection*{Completeness of  $\uparrow$ and $\downarrow$ w.r.t.\ \textbf{B} and \textbf{C}}%
We will assume the following definitions for contexts, the plugging
operator $\cdot\:[\:\cdot\:]$, and the trace function:
\begin{center}
  $\text{Context}\;Σ\coloneqq\Box\vsep Σ\prodl Γ\vsep Γ\prodr Σ$\\
  \begin{minipage}{0.45\linewidth}
    \begin{alignat*}{2}
      &\Box       \;&&[Γ']\mapsto Γ'\\
      &(Σ\prodl Γ)\;&&[Γ']\mapsto (Σ[Γ']\prod Γ)\\
      &(Γ\prodr Σ)\;&&[Γ']\mapsto (Γ\prod Σ[Γ'])
    \end{alignat*}
  \end{minipage}
  \begin{minipage}{0.45\linewidth}
    \begin{alignat*}{2}
      &\text{trace}(\Box)     \;&&\mapsto \mathbf{I}\\
      &\text{trace}(Σ\prodl Γ)\;&&\mapsto ((\mathbf{C}\prod \text{trace}(Σ))\prod Γ)\\
      &\text{trace}(Γ\prodr Σ)\;&&\mapsto ((\mathbf{B}\prod Γ)\prod \text{trace}(Σ))
    \end{alignat*}
  \end{minipage}
\end{center}
Given these definitions, we can show that the following rules for
quantifier raising are derivable:
\begin{center}
  \vspace*{0.5\baselineskip}
  \begin{pfbox}
    \AXC{$\text{trace}(Σ)\fCenter[A\himpr B]$} \AXC{$[C]\fCenter Δ$}
    \RightLabel{$\uparrow$}
    \BIC{$Σ[\cdot\mathbf{Q}(C\himpl(A\himpr B))\cdot]\fCenter Δ$}
  \end{pfbox}
  \begin{pfbox}
    \AXC{$Σ[\cdot A\cdot]\fCenter\cdot B\cdot$}
    \RightLabel{$\downarrow$}
    \UIC{$\text{trace}(Σ)\fCenter\cdot A\himpr B\cdot$}
  \end{pfbox}
  \vspace*{0.5\baselineskip}
\end{center}
And these rules can be combined to form one full quantifier movement,
reducing the type $\mathbf{Q}(C\himpl(A\himpr B))$ to $A$, while
changing the top-level type from $C$ to $B$:
\begin{center}
  \vspace*{0.5\baselineskip}
  \begin{pfbox}
    \AXC{$\vdots$}\noLine\UIC{$Σ[\cdot A\cdot]\fCenter\cdot B\cdot$}
    \RightLabel{$\downarrow$}
    \UIC{$\text{trace}(Σ)\fCenter\cdot A\himpr B\cdot$}
    \RightLabel{Foc$^R$} \UIC{$\text{trace}(Σ)\fCenter[A\himpr B]$}
    \AXC{$\vdots$}\noLine\UIC{$[C]\fCenter Δ$} \RightLabel{$\uparrow$}
    \BIC{$Σ[\cdot\mathbf{Q}(C\himpl(A\himpr B))\cdot]\fCenter Δ$}
  \end{pfbox}
  \vspace*{0.5\baselineskip}
\end{center}
We would like to show that, in fragment of the logic which is used for
natural language, the derived rules $\uparrow$ and $\downarrow$ are
complete w.r.t.\ the structural rules \textbf{B} and \textbf{C}. For
this purpose, we assume that:
\begin{itemize}
\item%
  there will be no occurrences of hollow structural connectives
  ($\!\!\himpl$,$\hp rod$,$\himpr$), \textbf{B} or \textbf{C} in
  the final sequent---the presence of these indicates unresolved
  movement, which means the sentence is not pronounceable;
\item%
  all occurrences of the quantifying licence \textbf{Q} will be of
  the form $\mathbf{Q}(C\himpl(A\himpr B))$.
\end{itemize}
Under these assumptions, we can derive that the only interesting
proofs which involve quantifiers will be of the form:
$$
L\mathbf{I}
    \ra\text{move up}
    \ra L\!\!{\himpl}
    \ra\text{auxiliary rules}
    \ra R{\himpr}
    \ra\text{move down}
    \ra\mathbf{I}^-
$$
There are three important facts to note here:
\begin{enumerate}
\item\label{no-axiom-BC}%
  \textbf{B}'s and \textbf{C}'s are structures, and therefore cannot
  be eliminated by axioms;
\item\label{cannot-overtake-Q}%
  during upwards or downwards movement, the quantifier is always
  attached to a hollow product, and the \textbf{B} and \textbf{C}
  rules only allow a quantifier to move past a \textit{solid} product;
  therefore, no quantifier can ever move past another quantifier;
\item%
  from \ref{no-axiom-BC} and \ref{cannot-overtake-Q}, we can derive
  that \textbf{B}'s and \textbf{C}'s introduced by upwards movement
  of a quantifier can only be eliminated by downwards movement
  \textit{of that same quantifier}.
\end{enumerate}
\note{%
  One move important fact to take note of is that the \textbf{B} and
  \textbf{C} are set up in such a way that every quantifier is forced
  to return to its original location. If the quantifier stops halfway
  through it's movement, there will be \textbf{B}'s and \textbf{C}'s
  left over in the sequent---and since \textbf{B}'s and \textbf{C}'s
  are always introduced in a positive context, and polarity is
  maintained, they can never be eliminated by an axiom.\\
  During upwards (or downwards) movement, the main connective on the
  left-hand side is always the hollow product. At every step, there
  are three possibilities:
  \begin{enumerate}
  \item Move further up (or down) using \textbf{B} and \textbf{C};
  \item Use residuation to focus on the left-hand side of the hollow
    product.\\
    During upwards movement, the left-hand side is necessarily a
    quantifier, and therefore the only applicable rule is
    $L\!\!\himpl$, which will eliminate the $\himpl$, and therefore
    stop the upwards movement.\\
    During downwards movement, the left-hand side is always a
    formula. Due to the fact that right-hand side will contain
    \textbf{B} and \textbf{C}, which can only be eliminated by
    downwards movement, there will be no way to finish the proof but
    through reversing the residuation, thereby forming a loop.
  \item Use residuation to focus on the right-hand side of the hollow
    product.\\
    However, any proof steps that work on the right-hand side of the
    hollow product will commute with upwards and downwards movement,
    and can therefore take place before upwards movement, or after
    downwards movement.
  \end{enumerate}%
  As for the interaction of multiple quantifiers, this is where the
  second modality (hollow) starts to play a role, as it ensures that
  no quantifier can ever move past a quantifier which is still
  moving, as the rules \textbf{B} and \textbf{C} only allow
  quantifiers to move past \textit{solid} products. This also entails
  that quantifiers can never switch places, as this would require one
  to move past the other.
}
\end{document}
