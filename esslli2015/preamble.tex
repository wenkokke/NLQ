\usepackage[usenames,dvipsnames]{color}
\usepackage[
  colorlinks = true,
  linkcolor  = Maroon,
  urlcolor   = MidnightBlue,
  citecolor  = ForestGreen
  ]{hyperref}
\usepackage{natbib}
\usepackage{ifthen}
\usepackage{enumitem}
\renewcommand{\hscodestyle}{\small}
%\usepackage{fullpage}


% use Wouter's setup for notes (extended with ``hidden'')
\newboolean{marginNotes}
\setboolean{marginNotes}{false}

\newboolean{showNotes}
\setboolean{showNotes}{true}

\newboolean{showHidden}
\setboolean{showHidden}{false}

\newcommand{\hidden}[1]{
  \ifthenelse{\boolean{showHidden}}
             {#1}
             {}}
\newcommand{\marginNote}[1]{
  \ifthenelse{\boolean{marginNotes}}
             {\marginpar{#1}}
             {#1}}
\newcommand{\todo}[1]{
  \ifthenelse{\boolean{showNotes}}
             {\marginNote{\textcolor{red}{\textbf{TODO:~}#1}}}
             {}}

\renewcommand{\sectionautorefname}{\S}
\renewcommand{\subsectionautorefname}{\S\S}

% use Buss' Proofs
\usepackage{mdframed}
\usepackage{bussproofs}
\EnableBpAbbreviations%
\def\fCenter{\ \vdash\ }
\def\limpl{\multimap}
\newenvironment{scprooftree}[1]%
  {\gdef\scalefactor{#1}\begin{center}\proofSkipAmount \leavevmode}%
  {\scalebox{\scalefactor}{\DisplayProof}\proofSkipAmount \end{center} }

% define various which I'll be sure to use symbols
\usepackage{textgreek}
\usepackage{stmaryrd}
\usepackage{scalerel}
\usepackage{pict2e}
\usepackage{picture}
\usepackage{ucs}
\usepackage[utf8x]{inputenc}

\newcommand{\varslash}{%
  \mathbin{\mathpalette\pictslash{{0}{1}}}%
}
\newcommand{\varbslash}{%
  \mathbin{\mathpalette\pictslash{{1}{-1}}}%
}

\makeatletter
\newcommand{\pictslash}[2]{%
  \vcenter{%
    \sbox0{$\m@th#1\varobslash$}\dimen0=.55\wd0
    \hbox to\wd 0{%
      \hfil\pictslash@aux#2\hfil
    }%
  }%
}
\newcommand{\pictslash@aux}[2]{%
    \begin{picture}(\dimen0,\dimen0)
    \roundcap
    \linethickness{.15ex}
    \put(0,#1\dimen0){\line(1,#2){\dimen0}}
    \end{picture}%
}
\newcommand{\pictholer}[1]{%
  \vcenter{%
    \sbox0{$#1$}\dimen0=.1\wd0
    \hbox to\dimen0{%
      \begin{picture}(\dimen0,\ht0)
      \linethickness{.15ex}
      \put(0,0.333\ht0){\line(1, 1){.18\ht0}}
      \put(0,0.666\ht0){\line(1,-1){.18\ht0}}
      \end{picture}%
    }%
  }%
}
\newcommand{\pictholel}[1]{%
  \vcenter{%
    \sbox0{$#1$}\dimen0=.1\wd0
    \hbox to\dimen0{%
      \begin{picture}(\dimen0,\ht0)
      \linethickness{.15ex}
      \put(0,0.333\ht0){\line(-1, 1){.18\ht0}}
      \put(0,0.666\ht0){\line(-1,-1){.18\ht0}}
      \end{picture}%
    }%
  }%
}
\makeatother

\newcommand{\holer}[1]{\hphantom{\pictholer{#1}}#1\pictholer{#1}}
\newcommand{\holel}[1]{\pictholel{#1}#1\hphantom{\pictholer{#1}}}

\DeclareUnicodeCharacter{8656}{$\varslash$}
\DeclareUnicodeCharacter{8658}{$\varbslash$}
\DeclareUnicodeCharacter{8666}{$\varoslash$}
\DeclareUnicodeCharacter{8667}{$\varobslash$}
\DeclareUnicodeCharacter{7478}{$$}% drop superscript J
\DeclareUnicodeCharacter{7480}{$^{L}$}
\DeclareUnicodeCharacter{7486}{$$}% drop superscript P
\DeclareUnicodeCharacter{7487}{$^{R}$}
\DeclareUnicodeCharacter{8314}{$^{+}$}
\DeclareUnicodeCharacter{8315}{$^{-}$}


% customise the appearance of itemise bullets
\def\myitem{\item[$\circ$]}

% customise the apprearance of lhs2TeX code blocks
\renewcommand\hscodestyle{%
   \setlength\leftskip{.5cm}%
   \footnotesize
}
